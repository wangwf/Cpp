\documentclass[11pt, oneside]{article}   	% use "amsart" instead of "article" for AMSLaTeX format
\usepackage{geometry}                		% See geometry.pdf to learn the layout options. There are lots.
\geometry{letterpaper}                   		% ... or a4paper or a5paper or ...
%\geometry{landscape}                		% Activate for for rotated page geometry
%\usepackage[parfill]{parskip}    		% Activate to begin paragraphs with an empty line rather than an indent
\usepackage{graphicx}				% Use pdf, png, jpg, or eps� with pdflatex; use eps in DVI mode
								% TeX will automatically convert eps --> pdf in pdflatex
\usepackage{amssymb}

\title{Social and Economic Network: Models and Analysis}
\author{Wenfeng Wang}
%\date{}							% Activate to display a given date or no date



\begin{document}
\maketitle
%\section{}
%\subsection{}



Here is a brief summary on  Social and Economic Networks: Model and Analysis.
Instructors: Mattew O. Jackson.

\section{Introduction}
\subsection{Motivation, why study networks?}

\begin{itemize}
\item Many economic, political, and social interactions are shaped by the local structure of relationships:
  \begin{itemize}
    \item trade of goods and services, most markets are not centralized...
    \item sharing of information, favors, risk, ...
    \item transmission of virus, opinions...
    \item access to info about jobs ...
    \item choices of behavior, education, ...
    \item political alliances, trade alliances ...
  \end{itemize}
\item Social network influence behavior
  \begin{itemize}
    \item crime, employment, human capital, voting, smoking, ...
    \item newtworks exhibit heterogeneity, but also have enough underlying structure to model
  \end{itemize}
\item Synthesisze
  \begin{itemize}
    \item Sociology
    \item Economics
    \item Computer Science
    \item Statistical Physics
    \item Math (random graph)...
  \end{itemize}
  Research interestings (Area of Research):
  \begin{itemize}
  \item Theory:\\
    Network formation, dynamics, design... ; how network influence behavior; coevulation?
  \item Empirical and experimental work:\\
    observe networks, patterns, influence; test theory and identify regularities.
  \item Methodology: \\
    how to measure and analyze networks
  \end{itemize}
\end{itemize}

Focus:
Random graph methods, Strategic, game theoretic techniques; hybrids, statistical models.

Models:
Provide insights into why we see certain phenomena: why short path length in social network? \\
Allow for comparative statis:
how does component structure change with density?\\
Predict out of sample: What will happen with a new policy (vaccine, R\&D subsidy,..)? \\
Allow for statistical estimation: Is there significant clustering on a local level or did it appear at random?\\


\subsection{Representing Networks}
$N=\{1,...,n\}$ nodes, vertices, agents, actors, players...
edges, links, ties: connections between nodes
\begin{itemize}
  \item may have intensity (weighted)
  \item may just be 0 or 1 (unweighted)
  \item may be ``undirected `` or ``directed''
\end{itemize}

{\it Adjacency Matrix, or adjacency list}
$g~in~\{0, 1\}^{n\times n}$, $g_{ij}$ indices a link, tie or edge between $i$ and $j$.
Network, $(N, g)$

Simplifying the complexity
\begin{itemize}
  \item Global patterns of networks:
    degree distribution, path lengths...
  \item Segregation Patterns
    node types and homophily
  \item Local Patterns
    Clustering, Transitivity, Support..
  \item Positions in networks
    Neighborhood, Centrality, influence...
\end{itemize}

path: a walk $(i_1, i_2, ..., i_k)$ with each node $i$ distinct.
Cycle: a walk where $i_i = i_k$
Geodesci: a shortest path between two nodes

\subsection{Components}
$(N,g)$ is connected if there is a path between every two nodes.

Component: maximal connected subgraph


Diameter,largest geodesic (largest shortest path, if unconnected, of the largest component)

 average Path length (less prone to outliers)

Neighborhood: $N_i(g) = \{j | ij~in~ g\}$ (usual convention $ii$ not in $g$)

Degree: $d_i =\# N_i(g)$

\subsection{Erdos-Renyi (1959, 1960) Random Graph}
\begin{itemize}
  \item Start with n nodes
  \item each link is formed independently with some probability $p$
  \item Serves as a benchmark ``G(n,p)''
\end{itemize}

dense, (almost) connected $d(n)\geq (1+\epilson) log(n), with \epilson\gt 0$

not complete IF $d(n)/n \rightarrow 0$

Theorem on Network structure:
IF $d(n)\geq (1+\epilson) log(n), with \epilson\gt 0$ and $d(n)/n \rightarrow 0$
{\it Then for large n, average path length and diameter are approcimately proportional to $log(n)/log(d)$.}
$$ \frac{AvgDist(n)} {log(n)/log(d)} \rightarrow 1 $$


Degree Distribution, G(n, p)

\begin{itemize}
\item probability that node has d links is binomial
  $\frac{(n-1)!}{d!(n-d-1)!} p^d(1-p)^{n-d-1}$
\item large $n$, small $p$, this is approximately a $Poisson$ distribution
  $\frac{(n-1)^d}{d!} e^{-(n-1)p}$ ``Poisson random graph''
\end{itemize}


Fat tails

\subsection{Clustering}

The fraction of my friends are friends of each other?

Clustering coefficients
$Cl_i(g) = \#\{ kj~in~g| k, j~in~N_i(g)\}/\#{kj|k,j in N_i(g)}$

Average clustering:
$Cl^{avg}(g) = \sum_i Cl_i(g)/n$

Overall clustering
$cl(g) = \sum_i \#\{ kj~in~g| k, j~in~N_i(g)\}/ \sum_i \#{kj|k,j in N_i(g)}$

\section{Background, Definitions, and Measures Continue}

\subsection{Homophily}
``Bird of a feather flock together'' -- Philemon Holland

age, race, gender, religion, profession... Reasons for Homophily
opportunity--contact theory;
benefits/costs;
social pressure;
social competition...

\subsection{Centrality Measures--Degree, Closeness, Decay, and Betweenness}
Position in Network.
Degree,
Clustering,
Distance to other nodes,
Centrality, influence, power...

\begin{itemize}
\item degree centrality $d/(n-1)$, --connectedness
\item Closeness, Decay -- ease of reaching other nodes
  $\frac{n-1}{ \sum_i l(i,j) }, ~~l(i,j)~distance (i,j)$
\item decay Centrality
  $C_i^d(g) = \sum_{j\neq i} \delta^{l(i,j)}$, $\delta \rightarrow 1$ becomes component size,  $\delta \rightarrow 0$ becomes degree.
$\delta$ in between decaying distance measure: weights distance exponentially.

Normalized decay centrality:  $C_i^d(g) = \sum_{j\neq i} \delta^{l(i,j)} / ((n-1)\delta)$, where $(n-1)\delta$ is the lowest decay possible.
\item Betweenness -- role as an intermediary, connector
  $P(i,j)$ number of geodesics between $(i,j)$
  $P_k(i,j)$ number of geodesics between $(i,j)$ that $k$ lies on.

  $$ \sum_{i,j\neq k} \fract{P_k(i,j)/P(i,j)}{ (n-1)(n-2)/2}  $$
\item Influence, Prestige, Egenvectors-- ``Not what you know, but who you know''

$$ C_i = a\sum_j g_{ij}C_j$$, centrality is proportional to the sum of neighbors' centralities
$C^e = a g C^e(g)$, where $C^e(g)$ is an eigenvector -- many possible solutions.

Google PageRankd: score of page is proportional to the sume of the scores of pages linked to it;
Random surfer model: start at some page on the web, randomly pick a link, follow it, repeat..

Bonacich Centrality:
$C^b(g) = a g 1 + b g ag1+ b^2 g^2 ag1+...
 = a(1-bg)^{-1}g1$

\item Diffusion Centrality $DC_i(p,T) = \sum_{t=1...T} (pg)^t 1$,
if T=1, proportional to degree.
if $p<1/\lambda_1$, T is large, becomes Katz-Bonacich.
if $p\ge 1/\lambda_1$, T is large, becomes eigenvector.

\end{itemzie}

\subsection{Random Networks Thresholds and Phase Transitions}

Theorem [Erdos and Reny 1959] A threshold function for the connectedness of a Poisson random network is $t(n) = log(n)/n$


\section{Random Networks}
\subsection{Growing random Network}
Citation networks, Web, Scientific network, Societies...

log-log distribution

Preferential attachment,  ``richer get richer''

Block model

Markov chain

MCMC

\section{Strategic Network Formation}

pairwise stability

Nash stability

\section{Diffusion on Networks}

Diffusion

\section{Learning on Networks}

\section{Games on Networks}

\end{document}
