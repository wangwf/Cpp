\documentclass[fleqn,twoside]{article}
\usepackage{graphicx}
\usepackage[figuresright]{rotating}


%\makeindex
\begin{document}

\title{Memorandum of ``Introduction to Recommender System''}

\author{ Wenfeng Wang \footnote{ Wenfeng.Wang@yahoo.com}\\
  Memorandum of Courera online course  ``Introduction to Recommender System".
  \\
 Instructors: Joseph A. Konstan, Michael Ekstrand
}
\date{\today}

\maketitle

\tableofcontents
%

\section{Introduction}

Learning Objectives
\begin{itemize}
 \item Understand what a recommender system
 \item history $\&$ background
\end{itemize}

Information Retrieval
\begin{itemize}
\item Static content base \\
     Invest time in indexing content
\item Dynamic information need \\
    Queries presented in ``real time"
\item Common approach: TFIDF
	\begin{itemize}
	  \item Rank documens by term overlap
	   \item Rank tems by frequency
	\end{itemize}
\end{itemize}

Information Filtering
\begin{itemize}
\item Reverse assumptions from IR  \\
	Static Information need \\
	Dynamic content base 
 \item Invest effort in modeling user need
	\begin{itemize}
	  \item  Hand-created profile
	  \item Machine learned profile
	  \item Feedback/updates
	\end{itemize}
 \item Pass new content through filters
\end{itemize}


Automated CF
C.F. Engine:  Ratings, Correlations;
Submit Rating;
Store Ratings;
compute Correlations;
Request Recommendations;
Identify Neighbors;
Select Items, Predict Rating.


User-User Collaborative Filtering

Recommenders

MovieLens  (Classic collaborative Filtering)
K-Nearest neighbor User-User


\subsection{Taxonomy of Recommender Systems}
Analytical Framework
\begin{itemize}
\item Dimensions of Analysis
	\begin{itemize}
	  \item Domain
	  \item Purpose
	  \item Recommendation Context
	  \item Whose Opinions
	  \item Personalization Level
	  \item Privacy and Trustworthiness
	  \item Interfaces
	  \item Recommendation Algorithms
	\end{itemize}
\item Domains of Recommendation
	 \begin{itemize}
	 \item Content to Commerce and Beyond
	    News, information, 	``text"
	   Products, vendors, bundles
		matchmaking (other people)
		Sequences (e.g., music playlists)
	\item One particularly interesing property
	  New items (e.g. moviews, books, ...)
	  Re-recommend old ones (e.g., groceries, music)
 	\end{itemize}

\item Purposes of Recommendation  \\
	Sales, Information \\
	Education of user/customer \\
	Build a community of users/customers around products of content \\
	e.g Tripadvisor.com; ReferralWeb

\item Context
	\begin{itemize}
		\item What is the User doing at the time of recommendation? \\
				Shopping, Listening to Music,  Hanging out with other people
		\item How does the context constrain the recommender?\\
			Groups, automatic consumption (vs. suggestions), Level of attention, level of interruption \\
			e.g.  pandora.com

	\end{itemize}
\item Whose Opinion?
		Experts?   \\
		Ordinary phoaks? \\
		People like you  \\
		e.g. wine.com;  phoaks.com; 
\item Personalization Level
	\begin{itemize}
		\item  Generic/ Non-Personalized  --- Everyone receives same recommendations
		\item Demographic \\	
			-- Matches a target group
		\item Ephemeral \\
			--Matches current activity
		\item Persistent \\
			-- Matches long-term interests
	\end{itemize}
\item Privacy and Trustworthineses
	\begin{itemize}
		\item Who knows what about me?
			Personal information revealed \\
			identity \\
			Deniability of preferences
		\item Is the recommendation honest?
			Biases built-in by operator-- business rules \\
			Vulnerability to external manipulation \\
			Transparency of ``recommenders"; Reputation
	\end{itemize}
\item Interfaces
	\begin{itemize}
	\item Types of Ouput: \\
		Predictions, Recommendations, Filtering, Organice Vs explicit presentation
	\item Types of Input\\
		Explicity; Implicit
	\end{itemize}

\item Recommendation Algorithms
	\begin{itemize}
	\item Non-Personalized Summary Statistics
	\item Content-Based Filtering
	   Information Filtering;  Knowledge-Based
	\item Collaborative Filtering
		User-User;  Item-Item;  Dimensionality Reduction
	\item Others \\
		Critique/Interview Based Recommendation;
		Hybrid Techniques.
	\end{itemize}
\end{itemize}

From the Abstract to the Specific
\begin{itemize}
\item Basic Model:  Users; Items; Ratings; (Community)

\item Non-Personalized Summary 

External Community Data: Best-seller, Most popular; Trending hot
Summary of Community Ratings: Best-liked
Examples:
	Zagat restaurant ratings; 
	Billboard music rankings;
	TripAdvisor hotel ratings

\item Content-based Filtering
	\begin{itemize}
	\item User Ratings $\times$ Item Attributs $\rightarrow$ Model
	\item Model applied to new items via attributes
	\item Alternative:  Knowledge-based \\
	  item attributes form model of item space
	   Users navigate/browse that space
	\item Examples:
		Personalized news feeds; 
		Artist or Genre music feeds
	\end{itemize}
\item Personalized Collaborative Filtering
	\begin{itemize}
	\item Use Opinions of others to predict/recommend
	\item User model -- set of ratings
	\item Item model -- set of ratings
	\item Common core: sparse matrix of ratings\\
		Fill in missing values (predict)
		Select promising cells (recommend)
	\item Several different techniques
	\end{itemize}
\item Collaborative Filtering Techniques
	\begin{itemize}
	  \item User-User\\
		 -- Select neighborhood of similar-taste people. (Variant: select people you know/trust) \\
		User their opinions
	\item Item-Item \\
		Pre-compute similarity among items via ratings
		Use own ratings to triangulate for recommendation
	\item Dimensionality reduction \\
		Intuition: taste yields a lower dimensionality.
		Compress and use a taste representation
	\end{itemize}
\item Note on Evaluation \\
	To properly understand relative merits of each approach:\\
		Accuracy of predictions.\\
		Userfulness of Recommendations \\
			Correctness, Non-obviousness; Diversity
		Computational performance
\item Other Approaches
	\begin{itemize}
	\item Interactive recommenders: Critique-based, dialog-based
	\item Hybrids of various techniques
	\end{itemize}
\end{itemize}


\section{Non-personalized recommenders}
\subsection{Learning Objectives}

Non-personalized averages can be effective in the right application
Association rule:
\begin{equation}
	\frac{P(X~~AND~~Y}{P(X)\times P(Y)}
\end{equation}

Product associations can provide useful non-personalized recommendations in a context --- Need to identify context; data source/scope.

\subsection{Preference and Ratings}

	We want to know: What do users like? \\
	What can observe? What Users tell us (ratings) and What users do (actions)?\\
	These are noisy measurements of preference:
		Explicit: Rating, Review, Vote; 
		Implicit:  Click, Purchase,  Follow
		
	Difficulties with Ratings
		Are ratings reliable and accurate?
		Do user preferences change?
		What does a rating mean?
		
\subsection{Predictions and Recommendation}
Often two come together. Prediction: 

\begin{itemize}
\item pro: helps quantify item.
\item con: provides something falsifiable
\end{itemize}

Recommendations
\begin{itemize}
\item pro:  provides good choices as a default
\item con: if perceived as top-n, can result in failure to explore (if top few seem poor)
\end{itemize}

\subsection{Scoring and Ranking}

\begin{itemize}
\item  Several ways of computing and displaying predictions
\item  How to rank items with sparse, time-shifting data
\item  Points in the design space for prediction and recommendation,  and tradeoff.
\end{itemize}

Displaying Aggregate Preference (Prediction) \\
Ranking Items (recommend)

\subsection{Displaying Aggregate Preference (Prediction)}

\begin{itemize}
\item Average rating/upvote proportion\\
	of people who vote, do they like it?
	Doesn't show popularity
\item Net upvotes / Number of likes\\
	Shows popularity
	No controversy
\item $\%\geq 4$ starts ('positive')
\item Full distribution 
\end{itemize}

{\it Goal of Display:  To help users decide to buy/read/view the item}

\subsection{Ranking Considerations}

Why not rank by score?
	Too little data.
	Score may be multivariate.
	Domain or business considerations:  old item, unfavored item
	
Ranking Considerations:\\
Confidence: How confident are we that this item is good?
Risk tolerance: High-risk, high-reward; Conservative recommendation
Domain and business consideration --Time ( Reddit: Old stories aren't interesting; eBay: items have short lifetimes)


Dumped means

Confidence Intervals


Scoring news stories:
Hacker News:
\begin{equation}
HackerNewsScore = \frac{(U-D-1)^{\alpha}}{(t_{now}- t_{post})^{\gamma}}\times P
\end{equation}

\begin{equation}
Reddit2010Score = \log_{10} max(1, |U-D|) + \frac{sign(U-D)t_{post}}{45000}
\end{equation}


\begin{itemize}
\item Sparsity, inconsistency, temporal concerns make data messy
\item Simple scoring doesn't necessarily match the domain or business
\item There are good ways to deal with this (decay time, penalties, damping
\end{itemize}



\section{Content-based Recommenders}

  Pure information filtering system
  Case-base reasoning systems
  Knowledge-base navigation system


Key concept: building a vector of attribute or keyword preferences

TFIDF

Case-based: \\
Structure a database of cases  around a set of relevant attributes (e.g. camera price, zoom, pixels) \\
Query based on an example or attribute query, and retrieve relevant cases \\
Open issue: Many ways to structure interaction

Knowledge-Based Recommender: \\
Case-Based Example with Navigation Interface. \\
FindMe System (e.g., Entree)



\section{User-User Collaborative Filtering}

\section{Evaluation}
Learning Objectives:
\begin{itemize}
\item Learning ways of evaluating the ``goodness" of a recommendation, and of a recommender algorithm or system:
	\begin{itemize}
	\item Accuracy metrics
	\item Error metrics
	\item Decision-support metrics
	\item User and Usage-centered metrics
	\end{itemize}

\item How predictions and recommendations (including top-n) are evaluated?
\item Retrospective and live approaches to evaluation
\end{itemize}

Early days:
\begin{itemize}
\item Accuracy and error measures:  MAE (Mean Absolute Error), RMSE, MSE
\item Decision-support metrics:  ROC AUC, Breese score, later precision/recall
\item Error meets decision-support/user experience:  Reversals
\item User-centered metrics:  Coverage, user retention, recommendation uptake, satisfaction
\end{itemize}

A commercial Look:
Lift, cross-sales, up-sales, conversions

More sophisticated top-n/rank metrics
Serendipity
Diversity

Theme 1: prediction vs. Top-N
\begin{itemize}
\item prediction is mostly about accuracy, possibly decision support; focused locally
\item Top-N is mostly about ranking, decision support; focused comparatively
\end{itemize}

Theme 2: More than just metrics

Theme 3: Unary Data

Theme 4:  Dead (Retrospective) vs Live Recs?

\subsection{Basic Accuracy Metrics}
MAE: Mean Absolute Error
MSE: Mean Squared Error
RMSE:  Root Mean squared error

Average user first, then item ? \\
OR average everything?

\subsection{Basic Decision Support Metrics}

   The concept of ``decision support" metrics:  Measure how well a recommender helps users make good decisions -- good decisions are about choosing ``good" items and avoiding ``bad" ones. 
   For recommendations, top of list is what matters most.
   
   Precision is the percentage of selected items that are ``relevant"  $P=\frac{N_{rs}}{N_s}$;
   Recall is the percentage of relevant items that are selected, $R=\frac{N_{rs}}{N_r}$ . 
   
 Goal:  Precision is about returning mostly useful stuff. Not waste user time; assumption is that there is more useful stuff than you want.
 	    Recall is about not missing useful stuff.  Not make a bad oversight; assumption is that you have time to filter through results to find the key result you need.
	    F-matrics :$F_1 =\frac{2 PR}{P+R}$
   
   Problems:
   \begin{itemize}
   \item Need ground truth;  interesting biases on rated items
   \item  Covers entire data set -- not targeted on top-recommended items
   $p @ n =  \frac{ N_{r@n} } {n}$.  
   \end{itemize}
   
   Mean Average Precision (MAP) averages over both multiple queries and over position in top-n retrieval. 
   \begin{eqnarray}{*}
   Map   &=& \frac{ \sum_{q=1}^{Q} Ave P(q)} {Q}  \\
   Ave P &=& \frac{\sum_{k=1}^{n} P(k) * rel(k)}{\# relevant~docs}
   \end{eqnarray}
   
   Receiver Operating Characteristic (ROC) is a plot of the performance of a classifier or filter at different thresholds. It plots true-positive against false positives:
   
   
\begin{itemize}
\item   Error rate and Reversals
\item Precision, Recall, MAP
\item Receiver operating characteristic
\end{itemize}

 $Precision@n $ and overall precision are perhaps the most widely used; ROC provides insight if the goal is to tune the recommender's use as a filter, or identify ``sweet spots" in its performance.

\subsection{Rank Metrics}
Measuring how good a recommender is at ranking.

\begin{itemize}
\item Prediction accuracy: how well does the recommender estimate preference?
\item Decision support: how well does the recommender do at finding good things?
\item Rank accuracy: how well does the recommender estimate relative preference? \\
	Putting items in order by preference
\end{itemize}

Mean Reciprocal Rank

Spearman Rank Correlation
Discounted Cumulative Gain

\subsection{Fallacy of Hidden Data Evaluation}
\subsection{More Metrics}

\section{Item-Item Collaborative Filtering}

Unary Data

\section{Dimensionality Reduction Recommenders}

\subsection{SVD}
\subsection{FunkSVD Training algorithm}
\subsection{Probabilistic Matrix Factorization}

\section{others}
\subsection{Threat Models}
\subsection{the Cold Start Problem}
\subsection{What wasn't covered}

\section{TBD}
Collaborative Filter
\begin{itemize}
\item Non-personalized 
\item content-based
\item user-item
\item user-user
\item item-item
\item knowledge-based
\item trusted-based
\item hybrid
\end{itemize}

\end{document}
